\chapter{Discusión}

\section{Interpretación de resultados}
Los resultados obtenidos en este estudio evidencian que la aplicación SpeechDown, al integrar APIs de inteligencia artificial generativa, puede facilitar el desarrollo del habla en niños con Síndrome de Down. La diferencia significativa en las puntuaciones del Grupo B sugiere que las actividades personalizadas y adaptadas mediante IA generan un impacto positivo en el proceso terapéutico. Además, la combinación de texto-a-voz y ejercicios interactivos contribuye a mejorar la experiencia de los usuarios, promoviendo una mayor motivación y adherencia a las terapias.

\section{Comparación con literatura previa}
Estos hallazgos coinciden con estudios previos que han demostrado el potencial de las tecnologías basadas en inteligencia artificial para apoyar procesos educativos y terapéuticos (por ejemplo, \cite{smith2021ai}, \cite{garcia2022speech}). Sin embargo, el enfoque particular en niños con Síndrome de Down y el uso de aplicaciones web responsive específicas para contextos latinoamericanos representa una aportación novedosa en la literatura, ampliando las posibilidades de acceso y personalización en terapias del habla.

\section{Limitaciones}
A pesar de los resultados prometedores, el estudio presenta algunas limitaciones. La muestra utilizada para las pruebas fue limitada y no representativa de la diversidad geográfica y socioeconómica de la población objetivo. Además, la dependencia de APIs externas puede implicar restricciones en costos y disponibilidad que afectan la escalabilidad del proyecto. Finalmente, se requiere un seguimiento a largo plazo para evaluar la efectividad sostenida de la aplicación en el desarrollo del habla.
