\chapter{Conclusiones}

\section{Conclusiones principales}
El desarrollo e implementación de la aplicación web SpeechDown ha demostrado ser una herramienta efectiva para apoyar el desarrollo del habla en niños con Síndrome de Down mediante la integración de inteligencia artificial generativa y tecnologías de texto a voz. Los resultados obtenidos evidencian que las actividades personalizadas generadas por IA, combinadas con una interfaz accesible y adaptativa, contribuyen positivamente al proceso terapéutico, mejorando la motivación y el progreso de los usuarios.

Además, la arquitectura full-stack utilizada facilita la escalabilidad, el mantenimiento y la integración de futuras mejoras, lo que posiciona a SpeechDown como una solución tecnológica innovadora para contextos terapéuticos latinoamericanos.

\section{Recomendaciones}
Se recomienda ampliar el estudio incorporando una muestra más amplia y diversa, que permita validar la efectividad y aceptación de la aplicación en diferentes contextos socioeconómicos y culturales. También es aconsejable integrar tecnologías complementarias como reconocimiento de voz y análisis en tiempo real para potenciar las capacidades interactivas de la aplicación.

Finalmente, se sugiere realizar evaluaciones longitudinales que permitan medir el impacto sostenido de SpeechDown en el desarrollo del habla y la calidad de vida de los niños usuarios, con el fin de ajustar y mejorar continuamente la herramienta.
